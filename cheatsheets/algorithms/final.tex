\documentclass[10pt,landscape,a4paper]{article}
\usepackage{multicol}
\usepackage{calc}
\usepackage{ifthen}
\usepackage[landscape]{geometry}
\usepackage{graphicx}
\usepackage{amsmath, amssymb, amsthm}
\usepackage{latexsym, marvosym}
\usepackage{pifont}
\usepackage{lscape}
\usepackage{graphicx}
\usepackage{array}
\usepackage{booktabs}
\usepackage[bottom]{footmisc}
\usepackage{tikz}
\usetikzlibrary{shapes}
\usepackage{pdfpages}
\usepackage{wrapfig}
\usepackage{enumitem}
\setlist[description]{leftmargin=0pt}
\usepackage{xfrac}
\usepackage{bbm}
\usepackage[pdftex,
            pdfauthor={William Chen},
            pdftitle={Probability Cheatsheet},
            pdfsubject={A cheatsheet pdf and reference guide originally made for Stat 110, Harvard's Introduction to Probability course. Formulas and equations for your statistics class.},
            pdfkeywords={probability} {statistics} {cheatsheet} {pdf} {cheat} {sheet} {formulas} {equations}
            ]{hyperref}
\usepackage[
            open,
            openlevel=2
            ]{bookmark}
\usepackage{relsize}
\usepackage{rotating}

 \newcommand\independent{\protect\mathpalette{\protect\independenT}{\perp}}
    \def\independenT#1#2{\mathrel{\setbox0\hbox{$#1#2$}%
    \copy0\kern-\wd0\mkern4mu\box0}}

\newcommand{\noin}{\noindent}
\newcommand{\logit}{\textrm{logit}}
\newcommand{\var}{\textrm{Var}}
\newcommand{\cov}{\textrm{Cov}}
\newcommand{\corr}{\textrm{Corr}}
\newcommand{\N}{\mathcal{N}}
\newcommand{\Bern}{\textrm{Bern}}
\newcommand{\Bin}{\textrm{Bin}}
\newcommand{\Beta}{\textrm{Beta}}
\newcommand{\Gam}{\textrm{Gamma}}
\newcommand{\Expo}{\textrm{Expo}}
\newcommand{\Pois}{\textrm{Pois}}
\newcommand{\Unif}{\textrm{Unif}}
\newcommand{\Geom}{\textrm{Geom}}
\newcommand{\NBin}{\textrm{NBin}}
\newcommand{\Hypergeometric}{\textrm{HGeom}}
\newcommand{\HGeom}{\textrm{HGeom}}
\newcommand{\Mult}{\textrm{Mult}}
\newcommand{\R}{\mathbb{R}}


\geometry{top=.2in,left=.2in,right=.2in,bottom=.2in}

\pagestyle{empty}
\makeatletter
\renewcommand{\section}{\@startsection{section}{1}{0mm}%
                                {-1ex plus -.5ex minus -.2ex}%
                                {0.5ex plus .2ex}%x
                                {\normalfont\large\bfseries}}
\renewcommand{\subsection}{\@startsection{subsection}{2}{0mm}%
                                {-1explus -.5ex minus -.2ex}%
                                {0.5ex plus .2ex}%
                                {\normalfont\normalsize\bfseries}}
\renewcommand{\subsubsection}{\@startsection{subsubsection}{3}{0mm}%
                                {-1ex plus -.5ex minus -.2ex}%
                                {1ex plus .2ex}%
                                {\normalfont\small\bfseries}}
\makeatother

\setcounter{secnumdepth}{0}
\graphicspath{ {./utils/images/} }
\setlength{\parindent}{0pt}
\setlength{\parskip}{0pt plus 0.5ex}

% -----------------------------------------------------------------------

\usepackage{titlesec}

\titleformat{\section}
{\color{blue}\normalfont\large\bfseries}
{\color{red}\thesection}{1em}{}
\titleformat{\subsection}
{\color{violet}\normalfont\normalsize\bfseries}
{\color{cyan}\thesection}{1em}{}
% Comment out the above 5 lines for black and white

\begin{document}

\raggedright
\footnotesize
\begin{multicols*}{3}

% multicol parameters
% These lengths are set only within the two main columns
%\setlength{\columnseprule}{0.25pt}
\setlength{\premulticols}{1pt}
\setlength{\postmulticols}{1pt}
\setlength{\multicolsep}{1pt}
\setlength{\columnsep}{2pt}

% Redefine bullet symbols for each level of itemize
\renewcommand{\labelitemi}{$\bullet$} % Level 1
\renewcommand{\labelitemii}{$\bullet$} % Level 2
\renewcommand{\labelitemiii}{$\bullet$} % Level 3
\renewcommand{\labelitemiv}{$\bullet$} % Level 4

\section{Graphs}

\subsection{Undirected Graphs}
\begin{itemize}
    \item $|E| = \frac{1}{2} \sum_{v \in V} deg(v)$
    \item The number of odd degree vertices is even
    \item If a graph is acyclic, there is a vertex of degree $\le 1$
\end{itemize}

\subsection{Trees}
For any graph $G = (V, E)$, if two of the following are true, then all three are and $G$ is a tree
\begin{itemize}
    \item $|E| = |V| - 1$
    \item $G$ is connected
    \item $G$ is acyclic
\end{itemize}

\subsection{Breadth-First Search}
\begin{itemize}
    \item $BFS(s)$ visits $v$ iff there is a path from $s \to v$
    \item Edges into then-undiscovered vertices define the BFS-tree of $G$
    \item Level $i$ of the BFS-tree contains all vertices with shortest path $i$ from the root
    \item \textbf{All non-tree edges} join vertices on the same, or adjacent levels of the tree
    \item If $G$ contains edges between vertices in the same layer, it is not bipartite, nor a tree
\end{itemize}

\subsection{Bipartite Graphs}
\begin{itemize}
    \item $G$ is bipartite iff you can partition $V$ into $V_1$ and $V_2$ such that all edges are between $V_1$ and $V_2$, i.e. no edges between vertices in different sets
    \item $G$ is bipartite iff $G$ has no odd cycles
\end{itemize}


\subsection{Depth First Search}
\begin{itemize}
    \item $DFS(s)$ visits $x$ iff there is a path from $s \to x$ (so you can find C.C.s)
    \item DFS Spanning Tree formed by edges into then-undiscovered vertices
    \begin{itemize}
        \item DFS tree not minimum depth, nor do its levels reflect the min distance
        \item Non-tree edges never join vertices on the same or adjacent level. Always join a vertex with one of its ancestors or descendants
        \item All vertices visited during $DFS(s)$ are a descendant of $s$ in the DFS tree
        \begin{itemize}
            \item For every edge $(u, v) \notin T_{DFS(s)}$, either $x$ is an ancestor of $y$ or $y$ is an ancestor of $x$
        \end{itemize}
    \end{itemize}
\end{itemize}

\subsection{Directed Acyclic Graphs}
\begin{itemize}
    \item \textbf{Source}: vertex with no incoming edges
    \item \textbf{Sink}: vertex with no outgoing edges
    \item Every DAG has a source and a sink
    \item Every DAG has a topological order, and every graph with a topological order is a DAG
\end{itemize}

\subsection{Topological Order}
An ordering of nodes $v_1, v_2, \ldots, v_n$ so that for every edge $(v_i, v_j)$, $i < j$.
\begin{itemize}
    \item To find, initialize map of in-degrees for each vertex, and a queue of vertices with in-degree 0.
    \item Then, while the queue isn't empty, remove a vertex, adding it to the ordering, and decrement its neighbors in-degree.
    \item If any of them become 0, add them to the queue.
\end{itemize}


\subsection{Cuts}
\begin{itemize}
    \item A cut of $G = (V, E)$ is a bipartition of $V$ into $S, V - S$ for some $S \subseteq V$
    \item $e = (u, v) \in (S, V - S)$ if exactly one of $u, v \in S$
    \item If $G$ is connected, then there is at least one edge in every cut
    \item Every cycle crosses a cut an even number of times
    \item \textit{Cut property}: Let $(S, V - S)$ be any cut, and let $e$ be the \textbf{min} cost edge with exactly \textbf{one} endpoint in $S$. Then \textbf{every} MST contains $e$.
    \item \textit{Cycle property}: Let $C$ be any cycle, and $f$ be the \textbf{max} cost edge belonging to $C$. Then \textbf{no} MST contains $f$.
\end{itemize}

\subsection{Minimum Spanning Tree}
\begin{itemize}
    \item \textbf{Algorithm}: $O(m\log(n))$
    \begin{itemize}
        \item Sort edges by increasing weight, initialize an empty tree $T$, and add each vertex to its own set.
        \item Then, for each edge $e = (u, v)$, if $u$ and $v$ are currently in different sets, add $e$ to $T$ and merge the sets containing $u$ and $v$.
    \end{itemize}
    \item \textbf{Proof}:
    \begin{itemize}
        \item Show that it is a tree
        \begin{itemize}
            \item Initially start with $|V| = n$ sets, and only add an edge if you are connecting two of them. Therefore, we end with $n - 1$ sets to add an edge between each original set
            \item Only add edges between disconnected components, so it must be acyclic, since each additional edge $e$ connecting $C_1$ and $C_2$ (two disconnected components) is the only edge between them. This means $C_1 + e + C_2$ has an odd number of edges in its cut, so there are no cycles formed.
        \end{itemize}
        \item Must be an MST
        \begin{itemize}
            \item Considered edges in increasing order of cost. Taking the first edge where the optimal and Kruskal's differ, we can exchange them for an equal or better solution.
        \end{itemize}
    \end{itemize}
\end{itemize}

\subsection{Disjoint Sets}
\begin{itemize}
    \item \textbf{Implementation}:
    \begin{itemize}
        \item Maintain a tree of pointers, where every vertex is labeled with the longest path ending at that vertex. To check set membership of $u$ and $v$, traverse to root and check if $root(u) = root(v)$. This is $O(\log(n))$
        \item To merge two sets, point root with smaller label to root with larger label, adjusting labels of the new root if necessary. This is $O(1)$.
    \end{itemize}
    \item \textbf{Properties}:
    \begin{itemize}
        \item If the label of a root is $k$, there are at least $2^k$ elements in the set.
    \end{itemize}
\end{itemize}

\section{Intervals}

\subsection{Scheduling the max number of intervals}
\begin{itemize}
    \item \textbf{Algorithm}: sort by finish time and select the next compatible interval
    \item \textbf{Proof}: Greedy stays ahead, by induction
    \begin{itemize}
        \item \textit{Claim}: Greedy algorithm is optimal
        \item \textbf{Lemma}
        \begin{itemize}
            \item $P(r)$: For greedy choices $g_1, \ldots, g_n$ and optimal choices $k_1, \ldots, k_m$, $f(g_r) \le f(k_r)$
            \item $P(1)$: $g_1$ is chosen to have the minimum finish time, so $f(g_1) \le f(k_1)$
            \item Suppose $P(r)$. Since $f(g_r) \le f(k_r) \le s(k_{r + 1})$, $k_{r + 1}$ is among the candidates considered for $g_{r + 1}$. Of those candidates, it picks the minimum finish time, so $f(g_{r + 1}) \le f(k_{r + 1})$.
        \end{itemize}
        \item By this lemma, we must have $n \ge m$, since since otherwise $k_{n + 1}$ is in the set of candidates for $g_{n + 1}$.
    \end{itemize}
\end{itemize}


\subsection{Partitioning intervals into the minimum number of sets}
\begin{itemize}
    \item \textbf{Algorithm}: sort intervals by start time, adding them to \textbf{any} compatible set. If no set is compatible, create a new one
    \item \textbf{Proof}: exploit structural property
    \begin{itemize}
        \item \textit{Claim}: greedy algorithm is optimal
        \item Let $d$ be the number of sets the greedy algorithm allocates. The $d$th set, $S_d$ is allocated because we had to assign some interval, $I_i$, that was not compatible with any of the $d - 1$ previous sets.
        \item Since we sorted by start time, all intervals $I_j \in S_1 \cup \ldots \cup S_{d - 1}$ have $s(I_i) \ge s(I_j)$. Thus, we have at least depth $d$ intervals, and so all valid partitions must have $\ge d$ sets.
    \end{itemize}
\end{itemize}

\section{Divide and Conquer}

\subsection{Master Theorem}
\begin{itemize}
    \item Given any recurrence of the form $T(n) = a T(\frac{n}{b}) + c n^k$ for all $n > b$, we have:
    \begin{itemize}
        \item If $a > b^k$, then $T(n) = \Theta(n^{\log_b a})$
        \item If $a < b^k$, then $T(n) = \Theta(n^k)$
        \item If $a = b^k$, then $T(n) = \Theta(n^k \log n)$
    \end{itemize}
\end{itemize}

\subsection{Root Finding}

Given a continuous function $f$ and two points $a < b$ such that $f(a) \cdot f(b) < 0$, there exists a root of $f$ in the interval $\left[ a, b \right]$ by the \textit{intermediate value theorem}. Since said root may be irrational, we aim to approximate it with an arbitrary precision $\epsilon$.

\begin{itemize}
    \item \textbf{Algorithm}: $Bisect(a, b, \epsilon)$
    \begin{itemize}
        \item If $b - a < \epsilon$, $a$ is a suitable approximation
        \item Otherwise, calculate the midpoint $m = (a + b)/2$
        \item If $f(m) \le 0$ then return $Bisect(m, b, \epsilon)$
        \item else return $Bisect(a, m, \epsilon)$
    \end{itemize}
    \item \textbf{Time}: $T(n) = T(\frac{n}{2}) + O(1) = O(\log(\frac{b - a}{\epsilon}))$
    \item \textbf{Proof}:
    \begin{itemize}
        \item $P(k) =$ For any $a, b$ such that $k\epsilon \le |a - b| \le (k + 1)\epsilon$, if $f(a)f(b) \le 0$, then we find an $\epsilon$ approx to a root using $\log k$ queries to $f$.
        \item $P(1)$: Output $a + \epsilon$, since the whole interval is at most $\epsilon$. This requires $0$ calls to $f$.
        \item Suppose $P(k)$ and consider an arbitrary $a$, $b$ s.t. $2k\epsilon \le |a - b| \le (2k + 1)\epsilon$.
        \item If $f(a + k\epsilon) = 0$, output $a + k\epsilon$.
        \item If $f(a)f(a + k\epsilon) < 0$, solve on the interval $\left[ a, a + k\epsilon \right]$. By I.H. this takes at most $\log(k)$ queries of $f$.
        \item Otherwise, we have $f(b)f(a + k\epsilon) < 0$, since $f(a)f(b) < 0$ and $f(a)f(a + k\epsilon) \ge 0$. Solve the interval $\left[ a + k\epsilon, b \right]$.
        \item In any case, we used at most $\log(k) + 1 = \log(2k)$ queries to $f$.
    \end{itemize}
\end{itemize}

\subsection{kth Smallest Element}

\begin{itemize}
    \item \textbf{Algorithm}: $f(S \in \mathbb{R}^n, k \in \mathbb{R})$
    \begin{itemize}
        \item Select an approximate median element $w$ using median of $\frac{n}{5}$ medians with subarrays of size $5$
        \item Partition each element into three sets, $S_{>}, S_{<}, S_{=}$
        \item If $k \le |S_{<}|$, recurse on $f(S_{<}, k)$
        \item Else, if $k \le |S_{<}| + |S_{=}|$, return $w$
        \item Else, recurse on $f(S_{>}, k - |S_{<}| - |S_{=}|)$
    \end{itemize}
\end{itemize}



\end{multicols*}
\end{document}
endsnippet
